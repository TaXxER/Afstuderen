\section{Literature study characteristics}
The literature research is performed by using the bibliographic databases Scopus and Web of Science with the following search query: \emph{("learning to rank" OR "learning-to-rank" OR "machine learned ranking") AND ("parallel" OR "distributed")}. An abstract based manual filtering step is applied where I filter those results that actually use the \emph{parallel} or \emph{distributed} terms in context to the \emph{learning to rank}, \emph{learning-to-rank} or \emph{machine learned ranking}. Studies focusing on efficient query evaluation instead of efficient model training are likely to meet all criteria listed. As a last step I will filter out studies based on the whole document that only focus on efficient query evaluation and not on parallel or distributed learning of ranking functions.\\

on Scopus the defined search query resulted in 65 documents. Only 14 of those documents used \emph{large scale}, \emph{parallel} or \emph{distributed} terms in context to the \emph{learning to rank}, \emph{learning-to-rank} or \emph{machine learned ranking}. 10 out of those 14 documents focussed on parallel or distributed learning of ranking functions.\\

The defined search query resulted in 16 documents on Web of Science. Four of those documents were also present in the set of 65 documents found using Scopus, leaving 61 unique documents. Only four of those 61 documents used \emph{large scale}, \emph{parallel} or \emph{distributed} terms in context to the \emph{learning to rank}, \emph{learning-to-rank} or \emph{machine learned ranking}, none of them focussed on parallel or distributed learning of ranking functions.\\

On Google Scholar the defined search query resulted in 3300 documents. We evaluate the first 300 search results.\\

Section \ref{sec:lit_descr} describe the approaches on parallel or distributed Learning-to-Rank found in literature.
\section{Literature description}
\label{sec:lit_descr}
Different approaches to parallelising Learning-to-Rank can be identified in literature. The following subsections describe the related work in the field categorised by approach.

\subsection{Parallelising Learning-to-Rank through ensemble learning}
Schapire proved in 1990 that a strong model can be generated by combining weak models through a procedure that he called boosting \cite{Schapire1990}. The invention of the boosting method resulted in an increasing focus in machine learning research on methods to combine multiple weak models, which are called ensemble learning methods. Well-known ensemble methods include gradient boosting \cite{Friedman2001}, bagging \cite{Breiman1996}, AdaBoost \cite{Freund1997} and stacked generalisation (stacking) \cite{Wolpert1992}.\\

Ensemble methods can be used to parallelise learning methods by training the weak models in the ensemble on different nodes. In the Learning-to-Rank field parallelisation of model training through ensemble learning is frequently achieved using decision tree learners and often using the boosting ensemble method, jointly called \acf{GBDT}. \ac{GBDT}'s are shown to be able to achieve good accuracy in a Learning-to-Rank setting when used in a pairwise \cite{Zheng2007} or listwise \cite{Chen2008} setting.

\subsubsection{Gradient Boosting}
Gradient Boosting \cite{Friedman2001} is an ensemble learning method in which multiple weak learners are iteratively added together into one ensemble model in such a way that new models focus more on those data instances that were misclassified before.\\

Kocsis et al. \cite{Kocsis2013} propose way to train multiple weak learners in parallel by extending those models that are likely to yield a good model when combined through boosting. Authors showed through theoretical analysis that the proposed algorithm asymptotically achieve equal performance to regular gradient boosting. \ac{GBDT} models could be trained in parallel using their BoostingTree algorithm.\\

Tyree et al. \cite{Tyree2011} describe a way of parallelising \ac{GBDT} models for Learning-to-Rank where the boosting step is still executed sequentially, instead the construction of the regression trees themselves are parallelised. The parallel decision tree building is based on Ben-Haim and Yom-Tov's work on parallel construction of decision trees for classification \cite{Ben-Haim2010}. Decision trees are build layer-by-layer where the calculations needed for building each new layer in the tree are divided over the nodes using a master-worker paradigm. The data is partitioned of the workers, who compress their share into histograms and send these to the master. The master uses those histograms to approximate the split and build the next layer. The master than communicates this new layer to the workers who can use this new layer to compute new histograms. this process is repeated until the tree depth limit is reached. The tree construction algorithm parallelised with described master-worker method is \ac{CART} \cite{Breiman1984}. Speed-up experiments on the LETOR and the Yahoo! Learning to Rank challenge data sets were performed. This parallel \ac{CART}-tree building approach showed speed-up of up to 42x on shared memory machines and up to 25x on distributed memory machines.\\

Ye et al. \cite{Ye2009} described how to implement the stochastic \ac{GBDT} model in a parallel manner using both MPI and Hadoop. Stochastic \ac{GBDT} differs from regular \ac{GBDT} models by using stochastic gradient boosting instead of regular gradient boosting. Stochastic gradient boosting is a minor alteration to regular gradient boosting that, at each iteration of the algorithm, trains a base learner on a sub-sample of the training set that is drawn at random without replacement \cite{Friedman2002}. Experiments showed the Hadoop implementation to result into too expensive communication cost to be useful. Authors believed that these high communication costs were a result of the communication intensive implementation that was not well suited for the MapReduce paradigm. The MPI approach proved to be successful and obtained near linear speed-ups.

\subsubsection{Boosting wrapped in Bagging}
Some approaches combine both boosting and bagging. Bagging \cite{Breiman1996}, also called bootstrap aggregating, is a ensemble learning method in which $m$ training sets $D_1..D_m$ are constructed from the training set $D$ by uniformly sampling data items from $D$ with replacement. The bagging method is parallelisable by training each $D_i$ with $i \in \{1..m\}$ on a different node.\\

Pavlov et al. \cite{Pavlov2010} from Yandex were the first to propose a boosting-wrapped-in-bagging approach, which they called BagBoo. The boosting step itself is not parallelisable, but the authors state that learning a short boosted sequence on a single node is still a do-able task.

Ganjisaffar \cite{Ganjisaffar2011b} proposed a pairwise boosting-wrapped-in-bagging model called BL-MART, in contrast to BagBoo which is a pointwise model. BL-MART adds a bagging step to the LambdaMART \cite{Wu2008} ranking model, a model that uses the Gradient Boosting \cite{Friedman2002} ensemble method combined with regression tree weak learners. In contract to BagBoo, BL-MART is limited in the number of trees. An experiment on the TD2004 dataset resulted in 4500 trees using BL-MART while 1.1 million trees were created with the BagBoo model.

\subsubsection{Stacked generalisation}
A \ac{DSN} is a processing architecture developed from the field of \emph{Deep Learning}, that uses the \ac{MSE} loss function that is easily optimisable. \ac{DSN} is based on the stacked generalization ensemble learning method \cite{Wolpert1992}, an approach where several learning models are stacked on top of each other in such a way that the output of a model is used as one of the input features of models higher up in the stack. Each layer in the stack learns has the task of learning the weights of the inputs of that layer. The close-form constraints between input and output weights allow the input weight matrices to be estimated in a parallel manner.\\

\subsection{Low complexity Learning-to-Rank}
Designing Learning-to-Rank algorithms with low time complexity for training is another approach towards large scale Learning-to-Rank. Pahikkala et al. \cite{Pahikkala2009} describe a pairwise \ac{RLS} type of ranking function, RankRLS, with time complexity $\mathcal{O}(n^3+n^2m)$, with $n$ the number of features and $m$ the number of training documents. The RankRLS ranking function showed ranking performance similar to RankSVM on the BioInfer corpus \cite{Pyysalo2007}, a corpus for information extraction in the biomedical domain.

\subsection{CCRank}
Wang et al. \cite{Wang2011a,Wang2011b} propose a parallel evolutionary algorithm based on \ac{CC}, a type of evolutionary algorithm. The \ac{CC} algorithm is capable of directly optimizing non-differentiable functions, as \ac{nDCG}, in contrary to many optimization algorithms.  the divide-and-conquer nature of the \ac{CC} algorithm enables parallelisation. CCRank showed an increase in both accuracy and efficiency on the LETOR 4.0 benchmark dataset compared to the baselines. It must be stated however that the increased efficiency was achieved through speed-up and not scale-up. Two reasons have been identified for not achieving linear scale-up with CCRank: 1) parallel execution is suspended after each generation to perform combination in order to produce the candidate solution, 2) Combination has to wait until all parallel tasks have finished, which may spend different running time.
\subsection{Parallel ListNet using Spark}
Shukla et al. \cite{Shukla2012} explored the parallelisation of the well-known ListNet Learning-to-Rank method using Spark. Spark is a parallel computing model that is designed for cyclic data flows which makes it more suitable for iterative algorithms. Spark is incorporated into Hadoop since Hadoop 2.0. The Spark implementation of ListNet showed near a linear training time reduction.
\subsection{nDCG-Annealing}
Karimzadeghan et al. \cite{Karimzadehgan2011} proposed a method using Simulated Annealing along with the Simplex method for its parameter search. This method directly optimises the often non-differentiable Learning-to-Rank evaluation metrics like \ac{nDCG} and \ac{MAP}. The authors successfully parallelised their method in the MapReduce paradigm using Hadoop. The approach showed to be effective on both the LETOR 3.0 dataset and their own dataset with contextual advertising data. Unfortunately their work does not directly report on the speed-up obtained by parallelising  with Hadoop, but it is mentioned that further work needs to be done to effectively leverage parallel execution.

\subsection{Distributed Stochastic Gradient Descent}
Long et al. \cite{Long2012} describe special case of their pairwise cross-domain factor Learning-to-Rank method using distributed optimization of \ac{SGD} based on Hadoop MapReduce. Parallelisation of the \ac{SGD} optimization algorithm was performed using the MapReduce based method described by  Zinkevich et al. \cite{Zinkevich2010} has been used. Real world data from Yahoo! has been used to show that the model is effective. Unfortunately the speed-up obtained by training their model in parallel is not reported.
\subsection{FPGA-based LambdaRank}
Yan et al. \cite{Yan2009,Yan2010,Yan2011,Yan2012} described the development and incremental improvement of a \ac{SIMD} architecture \ac{FPGA} designed to run the Neural Network based LambdaRank Learning-to-Rank algorithm. This architecture achieved a 29.3X speed-up compared to the software implementation when evaluated on data from a commercial search engine. The exploration of \ac{FPGA} for Learning-to-Rank showed other advantages of the \ac{FPGA} approach next to faster model training. In their latest publication \cite{Yan2012} the \ac{FPGA} based LambdaRank implementation showed it could achieve up to 19.52X power efficiency and 7.17X price efficiency for query processing compared to Intel Xeon servers currently used at the commercial search engine.
\subsection{GPGPU for Learning-to-Rank}
De Sousa et al. \cite{DeSousa2012} proposed a \ac{GPGPU} approach using the \ac{GPU} both learning the ranking function and for query processing, thereby improving both training time and query time. An association rule based Learning-to-Rank approach proposed by \cite{Veloso2008} has been implemented using the \ac{GPU} in such a way that the set of rules van be computed in parallel, in different threads, for each document. A speed-up of 127X in query processing time is reported based on evaluation on the LETOR dataset. The speed-up achieved at learning the ranking function was unfortunately not stated.

\subsection{Alternating Direction Method of Multipliers}
Duh et al. \cite{Duh2011} propose the use of \ac{ADMM} for the Learning-to-Rank task. \ac{ADMM} is a general optimization method that solves problems of the form
\begin{alignat*}{2}
\text{minimize }   &  f(x) + g(x) \\
\text{subject to } &  Ax + Bz = c
\end{alignat*}
by updating $x$ and $z$ in an alternating fashion. It holds the nice properties that it can be executed in parallel and that it allows for incremental model updates without full retraining. Duh et al. \cite{Duh2011} show how to use \ac{ADMM} to train a RankSVM model in parallel. Experiments showed the \ac{ADMM} implementation to achieve a 13.1x training time speed-up on 72 workers, compared to training on a single worker.\\

Another \ac{ADMM} Learning-to-Rank based approach was proposed by Boyd et al. \cite{Boyd2012}. Boyd et al. \cite{Boyd2012} implemented an \ac{ADMM} based Learning-to-Rank method in Pregel \cite{Malewicz2010}, a parallel computation framework for graph computations. No experimental results on parallelisation speed-up were reported on this Pregel based approach.
\subsection{Distributed Hyper-parameter Tuning using MapReduce}
Ganjisaffar et al. \cite{Ganjisaffar2011, Ganjisaffar2011b} observed that long training times are often a result of training many models for hyperparameter optimisation. Grid search is the \emph{de facto} standard of hyperparameter optimisation and is simply an exhaustive search through a manually specified subset of hyperparameter combinations. The authors show how to perform parallel grid search on MapReduce clusters, which is easy because grid search is an embarrassingly parallel method as hyperparameter combinations are mutually independent. They apply their grid search on MapReduce approach in a Learning-to-Rank setting to train a LambdaMART \cite{Wu2008} ranking model, a model that uses the Gradient Boosting \cite{Friedman2002} ensemble method combined with regression tree weak learners. Experiments showed that the solution scales linearly in number of hyperparameter combinations. However, the risk of overfitting grows overwhelmingly as the number of hyperparameter combinations grow, even when validation sets grow large.
\subsection{Distributed LambdaRank}
Svore and Burges \cite{Svore2010,Svore2012} designed two approaches to train the LambdaMART \cite{Wu2008} model in a distributed manner. Their first approach is applicable in case the whole training set fits in main memory on every node, in that case the tree split computations of the regressions trees in the LambdaMART tree ensemble are split and the data is not. The second approach distributes the training data and corresponding training computations and is therefore scalable to high amounts of training data. Both approaches achieves a six times speed-up over LambdaMART on 32 nodes compared to a single node.
\subsection{Learning-to-Rank with Bregman Divergences and Monotone Retargeting}
Acharyya et al. \cite{Acharyya2012} propose a Learning-to-Rank method searches for an order preserving transformation (\emph{monotone retargeting}) of the target scores that is easier for a regressor to fit. This approach is based on the observation that it is not necessary to fit scores exactly, since the evaluation is dependent on the order and not on the pointwise predictions themselves.\\

Bregman divergences are a family of distance-like functions that do not satisfy symmetry nor the triangle inequality. A well-known member of the class of Bregman divergences is Kullback-Leibler divergence, also known as \emph{information gain}.\\

Acharyya et al. \cite{Acharyya2012} defined a parallel algorithm that optimises a Bregman divergence function as surrogate of \ac{nDCG} that is claimed to be well suited for implementation of a \ac{GPGPU}. No experiments on speed-up have been performed.
\subsection{Parallel robust Learning-to-Rank}
Robust learning \cite{Huber1981} is defined as the task to lean a model in the presence of outliers. Yun et al. describe a \cite{Yun2014} robust Learning-to-Rank model called RoBiRank that has the additional advantage that it is executable in parallel. Their RoBiRank model uses parameter expansion to linearise a surrogate loss function, after which the elements of the linear combination can be divided over the available nodes. The Parameter expansion trick was proposed by Gopal and Yang \cite{Gopal2013}, who originally proposed it for multinomial logistic regression. Unfortunately, no speed-up experiments were mentioned for the RoBiRank method, since Yun et al. focussed their research more on robust ranking than on parallel ranking. The only reference to performance of RoBiRank in terms of speed is the statement that its training time on a computing cluster is comparable to the more efficient implementation by Lee and Lin \cite{Lee2014} of a linear kernel RankSVM \cite{Herbrich1999, Joachims2002}.