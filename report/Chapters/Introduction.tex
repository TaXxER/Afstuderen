\section{Motivation and Problem Statement}
\label{sec:motivation}
Ranking is a core problem in the field of information retrieval. The ranking task in information retrieval entails the ranking of candidate documents according to their relevance for a given query. Ranking has become a vital part of web search, where commercial search engines help users find their need in the extremely large document collection of the World Wide Web.\\

One can find useful applications of ranking in many application domains outside web search as well. For example, it plays a vital role in automatic document summarisation, where it can be used to rank sentences in a document according to their contribution to a summary of that document 
\cite{Bravo-Marquez2012}. Learning to Rank also plays a role in the fields of machine translation \cite{Hopkins2011}, automatic drug discovery \cite{Agarwal2010b}, the prediction of chemical reactions in the field of chemistry \cite{Kayala2011}, and it is used to determine the ideal order in a sequence of maintenance operations \cite{Rudin2009}. In addition, Learning to Rank has been found to be a better fit as an underlying technique compared to continuous scale regression-based prediction for applications in recommender systems \cite{Adomavicius2005,McNee2006}, like those found in Netflix or Amazon.\\

In the context of Learning to Rank applied to information retrieval, Luhn \cite{Luhn1957} was the first to propose a model that assigned relevance scores to documents given a query back in 1957. This started a transformation of the Information Retrieval field from a focus on the binary classification task of labelling documents as either \emph{relevant} or \emph{not relevant} into a \emph{ranked retrieval} task that aims at ranking the documents from most to least relevant. Research in the field of ranking models has long been based on manually designed ranking functions, such as the well-known BM25 model \cite{Robertson1994}, that simply rank documents based on the appearance of the query terms in these documents. The increasing amounts of potential training data have recently made it possible to leverage machine learning methods to obtain more effective and more accurate ranking models. Learning to Rank is the relatively new research area that covers the use of machine learning models for the ranking task.\\

In recent years several Learning to Rank benchmark data sets have been proposed that enable comparison of the performance of different Learning to Rank methods. Well-known benchmark data sets include the \emph{Yahoo! Learning to Rank Challenge} data set \cite{Chapelle2011a}, the \emph{Yandex Internet Mathematics competition}\footnote{http://imat-relpred.yandex.ru/en/}, and the \emph{LETOR} data sets \cite{Qin2010} that are published by Microsoft Research.\\

One of the concluding observations of the Yahoo! Learning to Rank Challenge was that almost all work in the Learning to Rank field focuses on ranking accuracy. Meanwhile, efficiency and scalability of Learning to Rank algorithms is still an underexposed research area that is likely to become more important in the near future as available data sets are rapidly increasing in size \cite{Chapelle2011b}. Liu \cite{Liu2007}, one of the members of the LETOR team at Microsoft, confirms the observation that efficiency and scalability of Learning to Rank methods has so far been an overlooked research area in his influential book on Learning to Rank.\\

Some research has been done in the area of parallel or distributed machine learning \cite{Chu2007,Chang2007}, with the aim to speed-up machine learning computation or to increase the size of the data sets that can be processed with machine learning techniques. However, almost none of these parallel or distributed machine learning studies target the Learning to Rank sub-field of machine learning. The field of efficient Learning to Rank has received some attention lately \cite{Asadi2013a,Asadi2013b,Busa-Fekete2012,Sousa2012,Shukla2012}, since Liu \cite{Liu2007} first stated its growing importance back in 2007. Only a few of these studies \cite{Sousa2012,Shukla2012} have explored the possibilities of efficient Learning to Rank through the use of parallel programming paradigms.\\

MapReduce \cite{Dean2004} is a parallel computing model that is inspired by the \emph{Map} and \emph{Reduce} functions that are commonly used in the field of functional programming. Since Google developed the MapReduce parallel programming framework back in 2004, it has grown to be the industry standard model for parallel programming. The release of Hadoop, an open-source version of MapReduce system that was already in use at Google, contributed greatly to MapReduce becoming the industry standard way of doing parallel computation.\\

Lin \cite{Lin2013} observed that algorithms that are of iterative nature, which most Learning to Rank algorithms are, are not amenable to the MapReduce framework. Lin argued that as a solution to the non-amenability of iterative algorithms to the MapReduce framework, iterative algorithms can often be replaced with non-iterative alternatives or by iterative alternatives that need fewer iterations, in such a way that its performance in a MapReduce setting is good enough. Alternative programming models are argued against by Lin, as they lack the critical mass as the data processing framework of choice and are as a result not worth their integration costs.\\

The appearance of benchmark data sets for Learning to Rank gave insight in the ranking accuracy of different Learning to Rank methods. As observed by Liu \cite{Liu2007} and the Yahoo! Learning to Rank Challenge team \cite{Chapelle2011b}, scalability of these Learning to Rank methods to large chunks of data is still an underexposed area of research. Up to now it remains unknown whether the Learning to Rank methods that perform well in terms of ranking accuracy also perform well in terms of scalability when they are used in a parallel manner using the MapReduce framework. This thesis aims to be an exploratory start in this little researched area of parallel Learning to Rank.\\

\section{Research Goals}
\label{sec:goals}
The objective of this thesis is to explore the speed-up in execution time of Learning to Rank algorithms through parallelisation using the MapReduce computational model. The set of Learning to Rank models described in literature is of such size that it is infeasible to conduct exhaustive experiments on all Learning to Rank models. Therefore, we set the scope of our scalability experiment to include those Learning to Rank algorithms that have shown leading performance on relevant benchmark data sets.\\

The existence of multiple benchmark data sets for Learning to Rank makes it non-trivial to determine the best Learning to Rank methods in terms of ranking accuracy. Given two ranking methods, there might be non-agreement between evaluation results on different benchmarks on which ranking method is more accurate. Furthermore, given two benchmark data sets, the sets of Learning to Rank methods that are evaluated on these benchmark data sets might not be identical. To be able to scope our experiment on the speed-up of the most accurate Learning to Rank methods we first need to identify what the most accurate Learning to Rank algorithms are. This brings us the the first research question: \\

\begin{description}
\item[RQ1] What are the best performing Learning to Rank algorithms in terms of ranking accuracy on relevant benchmark data sets?
\end{description}
\bigskip
Ranking accuracy is an ambiguous concept, as there exists several metrics that can be used to express the accuracy of a ranking. We will explore several metrics for ranking accuracy in section \ref{sec:how_to_evaluate_a_ranking}.\\

After determining the most accurate ranking methods, we perform speed-up experiment on distributed MapReduce implementations of those algorithms. We formulate this in the following research question: \\

\begin{description}
\item[RQ2] What is the speed-up of those Learning to Rank algorithms when executed using the MapReduce framework?
\end{description}
\bigskip
With multiple existing definitions of speed-up, we will use the speed-up definition known as \emph{relative speed-up} \cite{Sun1991}, which is formulated as follows:\\

$S_N = \frac{\text{execution time using one core}}{\text{execution time using \emph{N} cores}}$\\

The single core execution time in this formula is defined as the time that the fastest known single-machine implementation of the algorithm takes to perform the execution.

\section{Approach}
We will describe our research methodology on a per Research Question basis. Prior to describing the methodologies for answering the Research Questions, we will describe the characteristics of our search for related work.

\subsection{Literature Study Methodology}
A literature study will be performed to get insight in relevant existing techniques for large scale Learning to Rank. The literature study will be performed by using the following query:\\

\emph{("learning to rank" \emph{OR} "learning-to-rank" \emph{OR} "machine learned ranking") \emph{AND} ("parallel" \emph{OR} "distributed")}\\

and the following bibliographic databases:
\begin{itemize}
\item Scopus
\item Web of Science
\item Google Scholar
\end{itemize}

The query incorporates different ways of writing of Learning to Rank, with and without hyphens, and the synonymous term \emph{machine learned ranking} to increase search recall, i.e. to make sure that no relevant studies are missed. For the same reason the terms \emph{parallel} and \emph{distributed} are included in the search query. Even though \emph{parallel} and \emph{distributed} are not always synonymous, we are interested in both approaches in non-sequential data processing.\\

A one-level forward and backward reference search is used to find relevant papers missed so far. To handle the large volume of studies involved in the backward and forward reference search, relevance of the studies will be evaluated solely on the title of the study.

\subsection{Methodology for Research Question I}
To answer our first research question we will identify the Learning to Rank benchmark data sets that are used in literature to report the ranking accuracy of new Learning to Rank methods. These benchmark data sets will be identified by observing the data sets used in the papers found in the previously described literature study. Based on the benchmark data sets found, a literature search for papers will be performed and a cross-benchmark comparison method will be formulated. This literature search and cross-benchmark comparison procedure will be described in detail in section \ref{chap:cross_comparison}.

\subsection{Methodology for Research Question II}
\label{ssec:rq2_methodology}
To find an answer to the second research question, the Learning to Rank methods determined in the first research question will be implemented in the MapReduce framework and training time will be measured as a factor of the number of cluster nodes used to perform the computation. The HDInsight cloud-based MapReduce platform from Microsoft will be used to run the Learning to Rank algorithms on. HDInsight is based on the popular open source MapReduce implementation Hadoop\footnote{http://hadoop.apache.org/}.\\

To research the speed-up's dependence on the amount of data that is being processed, the training computations will be performed on data sets of varying sizes. We use the well-known benchmark collections LETOR 3.0, LETOR 4.0 and MSLR-WEB30/40K as a starting set of data sets for our experiments. Table \ref{tbl:initial_datasets} shows the data sizes of these data sets. The data sizes reported are not the total on-disk sizes of the data sets, but instead the size of the largest training set of all data folds (for an explanation of the concept of data folds, see \ref{sec:cross_validation}).

\begin{table}[!h]
\centering
\begin{tabular}{p{3.4cm}p{3.4cm}r}\toprule
Data set & Collection & Size \\
\midrule
OHSUMED     & LETOR 3.0       &   4.55 MB\\
MQ2008      & LETOR 4.0       &   5.93 MB\\
MQ2007      & LETOR 4.0       &  25.52 MB\\
MSLR-WEB10K & MSLR-WEB10K     & 938.01 MB\\
MSLR-WEB30K & MSLR-WEB30K     &   2.62 GB\\
\bottomrule
\end{tabular}
\caption{The LETOR 3.0, LETOR 4.0 and MSLR30/40K data sets and their data sizes}
\label{tbl:initial_datasets}
\end{table}

MSLR-WEB30K is the largest in data size of the benchmark data sets used in practice, but 2.62GB is still relatively small for MapReduce data processing. To test the how the computational performance of Learning to Rank algorithms both on cluster and on single-node computation scales to large quantities of data, larger data sets will be created by cloning the MSLR-WEB30K data set such that the cloned queries will hold new distinct query ID's.

\section{Thesis Overview}
\begin{description}
\item[Chapter 2: Background]{introduces the reader to the basic principles and recent work in the fields of Learning to Rank and the MapReduce computing model. Aims at bringing those unfamiliar with those topics up to speed to be able to understand the succeeding chapters of this thesis.}
\item[Chapter 3: Related Work]{concisely describes existing work in the field of parallel and distributed Learning to Rank.}
\item[Chapter 4: Benchmark Data Sets]{described the characteristics of the existing benchmark data sets in the Learning to Rank field.}
\item[Chapter 5: Cross Benchmark Comparison]{described the methodology of a comparison of ranking accuracy of Learning to Rank methods across benchmark data sets and described the results of this comparison.}
\item[Chapter 6: Selected Learning to Rank Methods]{describes the algorithms and details of the Learning to Rank methods selected in Chapter V.}
\item[Chapter 7: Implementation]{describes implementation details of the Learning to Rank algorithms in the Hadoop framework.}
\item[Chapter 8: MapReduce Experiments]{presents and discusses speed-up results for the implemented Learning to Rank methods.}
\item[Chapter 9: Conclusions]{summarizes the results and answers our research questions based on the results. The limitations of our research as well as future research directions in the field are mentioned here.}
\item[Chapter 10: Future Work]{describes several directions of research worthy follow-up research based on our findings.}
\end{description}