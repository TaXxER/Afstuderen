\chapter{Yahoo! Learning to Rank Challenge}
Yahoo's observation that all existing benchmark datasets were too small to draw reliable conclusions, especially in comparison with datasets used in commercial search engines, prompted Yahoo to release two internal datasets from Yahoo! search. The Yahoo! Learning to Rank Challenge\cite{Chapelle2011a} is a public Learning-to-Rank competition which took place from March to May 2010, with the goal to promote the datasets and encourage the research community to develop new Learning-to-Rank algorithms.\\

The Yahoo! Learning to Rank Challenge consists of two tracks that uses the two datasets respectively: a standard Learning-to-Rank track and a transfer learning track where the goal was to learn a specialized ranking function that can be used for a small country by leveraging a larger training set of another country. For this experiment I will only look at the standard Learning-to-Rank dataset because transfer learning is a separate research area that is not included in this thesis.
\section{Benchmark characteristics}
\begin{description}
\item 36k queries
\item 883k documents
\item 700 features
\end{description}

\section{Results}
\chapter{Yandex Internet Mathematics competition}
\section{Benchmark characteristics}
\section{Results}
\chapter{LETOR}
\section{Benchmark characteristics}
\section{Results}
\chapter{MSLR-WEB30k}
\section{Benchmark characteristics}
\section{Results}