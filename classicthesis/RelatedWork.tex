\chapter{Related Work}
\section{Search characteristics}
The literature research is performed by using the bibliographic databases Scopus and Web of Science with the following search query: \emph{("learning to rank" OR "learning-to-rank" OR "machine learned ranking") AND ("large scale" OR "parallel" OR "distributed")}. An abstract based manual filtering step is applied where I filter those results that actually use the \emph{large scale}, \emph{parallel} or \emph{distributed} terms in context to the \emph{learning to rank}, \emph{learning-to-rank} or \emph{machine learned ranking}. Studies focusing on efficient query evaluation instead of efficient model training are likely to meet all criteria listed. As a last step I will filter out studies focusing on efficient query evaluation.
\subsection{Scopus}
Defined search query resulted in 65 documents.
\subsection{Web of Science}
\section{CCRank}
Wang et al\cite{Wang2011} propose a parallel evolutionary algorithm based on \ac{CC}, a type of evolutionary algorithm. The \ac{CC} algorithm is capable of directly optimizing non-differentiable functions, as \ac{nDCG}, in contrary to many optimization algorithms.  In \ac{CC} algorithms to evolution process can be naturally parallelised 